% You need to select the XeTex or the LuaTeX engine to compile to pdf
% If you use Overleaf, you have to change the compiler (https://www.overleaf.com/learn/how-to/Changing_compiler)
% Reason: The common pdfTeX engine cannot use system fonts, and Overleaf has not even the MS Windows System fonts on their PCs it seems. Hence, we include a Aptos-font folder

\documentclass[aspectratio=169,xcolor=table,t]{beamer}

% Load package "amsmath" before PSI (because "fontspec" inside PSI should be loaded before "amsmath" according to an advice
\usepackage{amsmath, amssymb}
\usepackage[CEE, Red, blockcolored, PPTitemize]{PSI} % blockcolored is required if you want to have colored blocks, because the default beamer theme has uncolored boxes

%% My personal suggestions:
% Remove the "Figure" in front of a figure caption in beamer:
\setbeamertemplate{caption}{\raggedright\insertcaption\par}
% The advice to use non-serif fonts for math was from a time
% where screen-resolution was low. With HD and UHD, this is obsolete IMHO.
\usefonttheme[onlymath]{serif}


\title[Short title]{Add Title of\\Presentation}
\subtitle{Add Subtitle of\\Presentation}
\author[Author in Footnote]{Author on Title Page}
\date{28.06.2024}
\institute{Occation of Presentation}


\begin{document}

\begin{frame}[plain,c] % c = centered (or you have shift coordinates in PSI.sty)
 \titlepage
\end{frame}


\begin{frame}{Example slide title}

Formula in serif-font as in papers (can be changed in LaTeX preamble):
\begin{equation}
    a + b = c \int f(x)\,dx
\end{equation}

\begin{block}{Remember this!}
The block appearance can be easily changed in the LaTeX preamble
\end{block}

\begin{alertblock}{Very Important}
Sample text in red box
\end{alertblock}

\begin{exampleblock}{Top algined}
In the LaTeX preamble, we set [t] for top aligned content as in Powerpoint 
\end{exampleblock}

\end{frame}


\begin{frame}{Slide title}

Bullets are smaller than in LaTeX, and begin aligned with text: 
\begin{itemize}
 \item Level 1
 \begin{itemize}
     \item Level 2
     \begin{itemize}
          \item Level 3
      \end{itemize}    
 \end{itemize}     
\end{itemize}

\end{frame}



\begin{frame}{A list that is revealed}

  \begin{itemize}
  \item PSI-style 
  \item[--]  With dash instead of bullet if you like
    \begin{enumerate}
    \item Enumerated 1a
    \item Enumerated 1b
    \end{enumerate}
    \pause
  \item Item 2 (revealed on next slide)
    \pause
  \item Item 3
    \begin{itemize}
    \item Item 4
      \begin{itemize}
      \item Item 5
      \end{itemize}
    \end{itemize}
  \end{itemize}

\end{frame}



\begin{frame}{The PSI colors}

\textcolor{PSIgreen}{\rule{2cm}{1cm}}
\textcolor{PSIyellow}{\rule{2cm}{1cm}}
\textcolor{PSIpink}{\rule{2cm}{1cm}}
\textcolor{PSIblue}{\rule{2cm}{1cm}}
\textcolor{PSIred}{\rule{2cm}{1cm}}
\textcolor{PSIviolet}{\rule{2cm}{1cm}}

 \structure{This is a color predefined in beamer called \texttt{structure}}

\end{frame}

\begin{frame}[squeeze]{Titles over two lines\\ are possible}
  Some formulas, which can be tagged, e.g.\ \thetag{$\ast$}:
  \begin{align*}
    \Gamma  & = \int_a^b f(x)\,dx             && \tag{$\ast$}\\
    \sum_{n=1}^N x_{i_n} &= \Bigl(\frac{a+b}{c+d}\Bigr)\cdot \sin(x)  &&
  \end{align*}
  \begin{itemize}
  \item  $\alpha = \zeta$; \texttt{\textbackslash boldsymbol}: $\boldsymbol{\alpha} = \boldsymbol{\zeta}$ 
  \item $\mathcal{F}_t$, $\mathbb{R}^n$
  \item $\{t \mid t=1,\dots,T\}$
  \item $f\colon X\to Y$; $f\colon x\mapsto y$; use \texttt{\textbackslash colon}
  \item $y_nx_n^m$ versus $y_n^{}x_n^m$
  \end{itemize}
   \alert{The color of this text is called \texttt{alert} in beamer. We use `Squeeze' on this dense slide}
  
\end{frame}

\begin{frame}{Columns and Blocks}

\setbeamercolor{mycolor}{bg=gray!30}

You can start text high with a top-aligned frame in beamer

Columns:
  \begin{columns}
    \column[T]{.5\linewidth}
      \begin{block}{Title of block}
        Top-aligned block in first column
      \end{block}
      \begin{example}{Title of example block}
        Block in first column
      \end{example}
      Text outside of block in first column
      \vspace{1ex}
      
      \begin{beamercolorbox}[wd=\linewidth, colsep*=4pt]{mycolor}
         This is a box (without title)
       \end{beamercolorbox}
       
   \column[T]{.4\linewidth}
      \begin{block}{Title in block}
        Block in second column 
      \end{block}
      \begin{alertblock}{Title of alert block}
        Block in second column
      \end{alertblock}
      \begin{block}{}
        Block with empty title
      \end{block}
    \end{columns}
    
    \vspace{1ex}
    \begin{beamercolorbox}[wd=\linewidth, colsep*=4pt]{mycolor}
      This is a box (without title)
    \end{beamercolorbox}
 \end{frame}
 

 \begin{frame}[c]{Figures, Centering}
 A slide with vertical centering.
 
  \begin{figure}
    \includegraphics[width =.5\textwidth]{PSI_Logo_black}
    \caption{Figures are automatically horiz.\ centered in beamer}
    \label{fig:PSI}
  \end{figure}
  
\end{frame}

\begin{frame}{Overwrite the margins}
  You can overwrite the margins with a wide column:
  \setbeamercolor{mycolor}{fg=PSIblue, bg=structure!10}
\begin{columns}
  \column{1.1\linewidth}
    We need more space:
  \begin{beamercolorbox}[wd=\textwidth, dp=0.7\textheight, colsep*=4pt]{mycolor}
    This is a very long text that needs a lot of space this is a very long text that needs a lot of space this is a very long text that needs a lot of space
  \end{beamercolorbox}
\end{columns}
\end{frame}

\begin{frame}{Tables}

\begin{table}
\caption{Table looks like Word}
\arrayrulecolor{white} 
{\rowcolors{2}{PSIblue!25}{PSIblue!8} % start in second row with alternative scheme
\begin{tabular}{ |p{3cm}|p{3cm}|l|  }
\hline
\rowcolor{PSIblue}
\multicolumn{2}{|c|}{\textcolor{white}{Country List}} & \textcolor{white}{Name} \\
\hline
Afghanistan & AF &AFG \\
Aland Islands & AX   & ALA \\
Albania &AL & ALB \\
Algeria    &DZ & DZA \\
\hline
\end{tabular}}
\end{table}

\begin{table}
\caption{Table looks like Word}
\arrayrulecolor{white} 
{\rowcolors{2}{PSIred!25}{PSIred!8} % start in second row with alternative scheme
\begin{tabular}{ |p{3cm}|p{3cm}|l|  }
\hline
\rowcolor{PSIred}
\multicolumn{2}{|c|}{\textcolor{white}{Country List}} & \textcolor{white}{Name} \\
\hline
Afghanistan & AF &AFG \\
Aland Islands & AX   & ALA \\
Albania &AL & ALB \\
Algeria    &DZ & DZA \\
\hline
\end{tabular}}
\end{table}

\end{frame}

\end{document}
