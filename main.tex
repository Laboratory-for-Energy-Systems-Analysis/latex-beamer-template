% Compile with LuaTex or XeTex, because you need the Aptos-Font
% If you use Overleaf, you have to change the compiler (https://www.overleaf.com/learn/how-to/Changing_compiler)

% The folder "Aptos"  with the Aptos-font is required if you do not use a Windows PC, for example by using the online LaTeX-editor Overleaf.

% Things where you have to adapt are with %%%

\documentclass[aspectratio=169,t]{beamer}  % t = top aligned text on slides as in Powerpoint

\usepackage{graphicx} 
\usepackage{amsmath, amssymb}
\usepackage{xcolor} % only required for convenience RGB color definitions further below: [0,255] as in Powerpoint
\usepackage[no-math]{fontspec}

\setsansfont[Path = ./Aptos/,
UprightFont = Aptos,
BoldFont = Aptos-Bold,
ItalicFont = Aptos-Italic,
BoldItalicFont = Aptos-Bold-Italic,
Extension = .ttf]{aptos}

%%% Convenience definition: The center name is used  just only once, in the footnote coming next
\newcommand{\PSIcenter}{PSI Center for Energy and Environmental Sciences}

\setbeamertemplate{footline}[text line]{%
  \parbox{\linewidth}{\vspace*{-8pt}\color{gray}\insertpagenumber\hfill\insertshortauthor:\ \insertshorttitle\hfill\PSIcenter\hfill \insertdate
}}

\setbeamertemplate{navigation symbols}{}

%%% if you use another cloud not found here, copy the corresponding picture out from the Powerpoint-Master-slide:
\pgfdeclareimage[height=\paperheight]{mybackground}{%
PSI_Cloud_Blue}
%PSI_Cloud_Red}
%PSI_Cloud_DarkBlue}

%%% if you use another center not found here, copy the corresponding picture out from the Powerpoint-Master-slide:
\pgfdeclareimage[height=0.5cm]{myCenter}{%
Center_CEE_White}
%Center_CAS_White}
%Center_CCS_White}
%Center_CLS_White}
%Center_CNM_White}
%Center_CPS_White}
%Center_NES_White}
%Center_SCD_White}

% The next two are the same pictures for all centers:
\pgfdeclareimage[height=0.5cm]{mySlideLogo}{PSI_Logo_black}
\pgfdeclareimage[height=0.8cm]{myLogo}{PSI_Logo_White}

\setbeamercolor{title}{fg=white}
\setbeamerfont{title}{size=\fontsize{22}{24}\selectfont}
\setbeamerfont{author}{size=\fontsize{9}{12}\selectfont}
\setbeamerfont{institute}{size=\fontsize{9}{12}\selectfont}
\setbeamerfont{date}{size=\fontsize{9}{12}\selectfont}
\setbeamertemplate{frametitle}{\vspace*{0.7ex}\color{black}\bfseries\insertframetitle}

\setbeamertemplate{title page}{
        \begin{picture}(0,0)
            \put(-28,-154.3){%
                \pgfuseimage{mybackground}}
                
             \put(0, 70){%
                \pgfuseimage{myLogo}}
                
             \put(60, 74){%
                \pgfuseimage{myCenter}}
                
            \put(0,-110.7){%
                \begin{minipage}[b][45mm][t]{226mm}
                   \begin{beamercolorbox}[wd=0.9\textwidth,sep=1em]{title}
                       \usebeamerfont{title}\textbf{\inserttitle}\vspace*{0.5ex}\par
                       \usebeamerfont{subtitle}{\textbf{\insertsubtitle}}
                    \end{beamercolorbox}       
               \end{minipage}}

             \put(0,-240){%
                \begin{minipage}[b][45mm][t]{226mm}
                   \begin{beamercolorbox}[wd=0.9\textwidth,sep=1em]{title}
                       \usebeamerfont{author}{\insertauthor\par}
                        \usebeamerfont{institute}{\insertinstitute},
                        \usebeamerfont{date}{\insertdate}  
                    \end{beamercolorbox}       
               \end{minipage}}

            \end{picture}
}

\logo{\pgfuseimage{mySlideLogo}\hspace{22pt}\vspace{225pt}}

% make the bullet smaller:
\setbeamertemplate{itemize item}{\small\raise0.5pt\hbox{\color{black}\textbullet}}
\setbeamertemplate{itemize subitem}{\small\raise0.5pt\hbox{\color{black}\textbullet}}
\setbeamertemplate{itemize subsubitem}{--}
% keep subitems at normal size:
\setbeamertemplate{itemize/enumerate subbody begin}{\normalsize}
\setbeamertemplate{itemize/enumerate subsubbody begin}{\normalsize}
% remove first-level indent:
\settowidth{\leftmargini}{\usebeamertemplate{itemize item}}
\addtolength{\leftmargini}{\labelsep}

% some colors for convenience
\definecolor{PSIblue}{RGB}{0,20,230}
\definecolor{PSIgreen}{RGB}{0,240,160}
\definecolor{PSIred}{RGB}{220,0,90}
\definecolor{PSIviolet}{RGB}{110,20,220}
\definecolor{PSIyellow}{RGB}{240,245,0}
\definecolor{PSIpink}{RGB}{240,80,250}

% modify blocks to your likening 
\setbeamertemplate{blocks}[rounded][shadow=false]
\setbeamercolor{block title}{bg=PSIblue,fg=white}
\setbeamercolor{block body}{bg=PSIblue!10,fg=black}
\setbeamercolor{block title example}{bg=PSIgreen,fg=black}
\setbeamercolor{block body example}{bg=PSIgreen!10,fg=black}
\setbeamercolor{block title alerted}{bg=PSIred,fg=white}
\setbeamercolor{block body alerted}{bg=PSIred!10,fg=black}
\setbeamercolor{structure}{fg=PSIblue}
\setbeamercolor{alert}{fg=PSIred}

% Disable shading between block title and block content
\makeatletter
\pgfdeclareverticalshading[lower.bg,upper.bg]{bmb@transition}{200cm}{color(0pt)=(lower.bg); color(4pt)=(lower.bg); color(4pt)=(upper.bg)}
\makeatother

%% My personal suggestions:

% Remove the silly "Figure" in front of a figure caption in beamer:
\setbeamertemplate{caption}{\raggedright\insertcaption\par}

% The advice to use non-serif fonts for math was from a time
% where screen-resolution was low. With HD and UHD, this is obsolete IMHO.
\usefonttheme[onlymath]{serif}
% The math spacing is wrong. We have to reset the letters font to T1 
\AtBeginDocument{\DeclareSymbolFont{pureletters}{T1}{lmr}{\mddefault}{it}}


%%% Start of presentation

\title[Large prosumers: Vertical integration and storage]{Large prosumers under \\ capacity constraints}
\subtitle{Market power effects of vertical integration and of storage}
\author[Wan, Densing, McKenna]{Yi Wan, \underline{Martin Densing}, Russell McKenna}
\date{2.7.2024}
\institute{EURO 2024, Copenhagen}


\begin{document}

\begin{frame}[plain,c] % c = centered (or you have shift coordinates in above code)
 \titlepage
\end{frame}


\begin{frame}{Motivation}
\textbf{New technologies in electricity markets may lead to new forms of deman- market power: 1.~Power-to-Gas (main part of talk), 2.~Storage}
\begin{itemize}
\item Demand-side related issues of market power were treated in literature (separately):
\vspace{-2ex}
\begin{itemize} 
    \item Vertical integration (consumption and production) 
    \item Monopsony (demand-side market power)
    \item Nash-Cournot equilibrium under capacity constraints  
\end{itemize}  
\end{itemize}
\medskip
\begin{itemize}
\item Numerical analyses exist, e.g.:
\begin{quote}
\item  "The largest social gain is obtained when PtG is operated independently."
\end{quote}
[Megy \& Massol (2023), \textit{Is Power-to-Gas always beneficial? The implications of ownership structure}]
\end{itemize}
\medskip
 $\rightarrow$ Missing (as far as we know): Analysis of \structure{closed-form solution of a producer and a consumer under market power and under capacity constraints}
\end{frame}

\begin{frame}{Profit maximization}
\begin{itemize}
\item 2 settings: One integrated player, and two separated players.
\item Other `players' are assumed to be represented by  a linear price-demand curve:
\begin{itemize}
\item $p = p(Q) = p_0 - \beta Q$, net-productions: $Q = q - q_r$ 
\end{itemize} 
 \item $q$: production, $q_r:$ `retail'-consumption, e.g.\ by power-to-gas
\end{itemize}

\begin{columns}
    \column[T]{.6\linewidth}
       \structure{\textbf{\underline{Integrated player}}}\vspace*{-2ex}
       \begin{align*}
       \max\ & (p-c)q + (\alpha-p)q_r \notag \\
            \text{s.t.} & \left\{
                \begin{aligned}
                    0 \leq & q \leq X, \\
                    0 \leq & q_r \leq X_r,
                  \end{aligned}
            \right. \\
        \frac{dp}{dq} & = - \frac{dp}{dq} := - \theta\beta     
        \end{align*}
        \vspace{-3.8ex}
        \begin{itemize}
       \item  $\theta, \theta_r\in[0,1]$: conjectural variation parameters
       \item  $c$: constant marginal production cost
       \item $\alpha$: fixed selling price of consumption (e.g.\ via H$_2$)
       \end{itemize}
    \column[T]{.4\linewidth}  
       \structure{\textbf{\underline{Separated players}}}\vspace*{-2ex}
    \begin{align*}
        \text{\structure{Producer: }}&& \max\ & p q_{} - c q_{}  \notag \\
        && \text{s.t.}\ & 
        \begin{aligned}
            & 0 \leq q \leq X,          
        \end{aligned} \\
       && \frac{dp}{dq} & := - \theta\beta \\
       \text{\structure{Consumer: }}&&  \max\ &  \alpha q_r - p q_r  \notag \\
           && \text{s.t.}\ & 
            \begin{aligned}
           & 0 \leq q_r \leq X_r.         
            \end{aligned} \\
         && \frac{dp}{dq} & := \theta_r\beta 
        \end{align*}
\end{columns}

\end{frame}

\begin{frame}{Integrated player: Optimal $q$ and $q_r$}
\vspace*{-2ex}
\renewcommand{\arraystretch}{0.8}
\begin{table}
\scriptsize
\centering
$
\begin{array}{c|c|c|lc}
    \hline
    \textbf{Case} & \mathbf{q}  & \mathbf{ q_r} & \textbf{Condition}  \\
    \hline
    %%% Case q=0, q_r=0:
    % ------------------------------
    1&0 & 0 & \alpha \leq p_0 \leq c  \\
    \hline
    %%% Case q>0, q_r=0:  
    % -----------------------------
    2&\frac{1}{1+\theta} \frac{1}{\beta} (p_0-c)  & 0 & \alpha \leq  c \\
    &&& 0 < \frac{1}{1+\theta} \frac{1}{\beta} (p_0-c) < X\ (\Rightarrow p_0>c) \\
   \hline
    %%% Case q=0, q_r>0:
    % --------------------
     3&0 &  \frac{1}{1+\theta} \frac{1}{\beta} (\alpha - p_0) & c \geq \alpha  \\
    &&& 0<\frac{1}{1+\theta} \frac{1}{\beta} (\alpha - p_0) < X_r\ (\Rightarrow \alpha>p_0) \\ 
    \hline
   %%% Case q=X, q_r=0:
   % ---------------------------   
   4&X  & 0 & X \leq \frac{1}{1+\theta} \frac{1}{\beta} (p_0-c) \ (\Rightarrow p_0>c)  \\
   &&& X \leq \frac{1}{1+\theta} \frac{1}{\beta} (p_0-\alpha) \ (\Rightarrow p_0>\alpha)  \\
   \hline
    %%% Case q=0, q_r=X_r:
    % --------------------
     5&0 & X_r &  X_r \leq \frac{1}{1+\theta} \frac{1}{\beta} (c -p_0)  \ (\Rightarrow c \geq p_0) \\ 
    &&& X_r \leq \frac{1}{1+\theta} \frac{1}{\beta} (\alpha - p_0)\ (\Rightarrow \alpha\geq p_0)  \\
   \hline
    %%% Case q>0, q_r>0:
    %---------------------------
    6&q\ \text{(free)}  & q - \frac{1}{1+\theta}\frac{1}{\beta}(p_0-c)
        & c = \alpha \\
    &&& 0 < q < X  \\
    &&& \frac{1}{1+\theta}\frac{1}{\beta}(p_0-c) < q  < X_r + \frac{1}{1+\theta}\frac{1}{\beta}(p_0-c) \\
   \hline
    %%% Case q=X, q_r>0:
    % ----------------------------
    7&X  & X - \frac{1}{1+\theta} \frac{1}{\beta} (p_0 - \alpha)
           & \alpha \geq c \\
    &&& \frac{1}{1+\theta} \frac{1}{\beta} (p_0-\alpha) < X < X_r + \frac{1}{1+\theta} \frac{1}{\beta} (p_0-\alpha) \\
   \hline
    %%% Case q>0, q_r=X_r: 
    % ------------------------
    8&X_r + \frac{1}{1+\theta} \frac{1}{\beta} (p_0-c) & X_r &  \alpha \geq c \\
    &&& 0 < X_r + \frac{1}{1+\theta} \frac{1}{\beta} (p_0-c) < X \\
   \hline
    %%% Case q =X, q_r =X_r: 
    % -------------------------
    9&X  & X_r & \alpha \geq c \\
    &&& X \leq X_r + \frac{1}{1+\theta} \frac{1}{\beta} (p_0-c) \\
    &&& X \geq X_r + \frac{1}{1+\theta} \frac{1}{\beta} (p_0-\alpha) \\
   \hline   
 \end{array}
 $
\end{table}
\renewcommand{\arraystretch}{1}
\end{frame}

\begin{frame}{Separated players: Optimal $q$ and $q_r$}
\vspace*{-3ex}
\renewcommand{\arraystretch}{0.5}
\tiny
\begin{table}
\makebox[1.2\textwidth][c]{
$
\begin{array}{c|c|c|lc}
   \hline
  \textbf{Case} &\mathbf{q} & \mathbf{ q_r} & \textbf{Condition } \\
   \hline
%%% Case q=0, q_r=0:   
%-------------------
   1&0 & 0  & \alpha \leq p_0 \leq  c  \\
   \hline
%%% Case q>0, q_r=0: 
%---------------------
   2&\frac{1}{1+\theta} \frac{1}{\beta} (p_0-c)  & 0 & c \geq \alpha - \theta(p_0-\alpha) \\
    &&& 0 < \frac{1}{1+\theta} \frac{1}{\beta} (p_0-c) < X\quad(\Rightarrow c<p_0\overset{(q_r)}{\Rightarrow}\alpha<p_0) \\
   \hline
%%% Case q=0, q_r>0:
% --------------------
     3&0 &  \frac{1}{1+\theta_r} \frac{1}{\beta} (\alpha - p_0) & c \geq \alpha + \theta_r(p_0-c)  \\
    &&& 0<\frac{1}{1+\theta_r} \frac{1}{\beta} (\alpha - p_0) < X_r \ (\Rightarrow \alpha>p_0 \Rightarrow c>p_0) \\
   \hline
%% Case  q=X, q_r>0:
% --------------------
   4&X & 0 & X \leq \frac{1}{1+\theta} \frac{1}{\beta} (p_0-c) \quad(\Rightarrow c<p_0)  \\
   &&& X \leq \frac{1}{\beta} (p_0-\alpha)\quad(\Rightarrow \alpha<p_0) \\
   \hline
    %%% Case q=0, q_r=X_r:
    % --------------------
     5&0 & X_r & X_r \leq  \frac{1}{\beta} (c -p_0)  \ (\Rightarrow c \geq p_0) \\
    &&& X_r \leq \frac{1}{1+\theta_r} \frac{1}{\beta} (\alpha - p_0) \ (\Rightarrow \alpha \geq p_0) \\
   \hline
%%% Case q>0, q_r>0:
% ----------------------
   6&\frac{1}{\theta\theta_r + \theta + \theta_r} \frac{1}{\beta} \bigl(\alpha - c + \theta_r (p_0-c) \bigr)
    & \frac{1}{\theta\theta_r + \theta + \theta_r} \frac{1}{\beta}\bigl(\alpha-c-\theta(p_0-\alpha)\bigr) 
    &  \text{Market power: }\neg(\theta=\theta_r=0): \\
    &&& 0 <  \frac{1}{\theta\theta_r + \theta + \theta_r} \frac{1}{\beta} \bigl(\alpha - c + \theta_r (p_0-c) \bigr) < X\ (\Rightarrow \alpha>c-\theta_r(p_0-c))\\
    && & 0 < \frac{1}{\theta\theta_r + \theta + \theta_r} \frac{1}{\beta}\bigl(\alpha-c-\theta(p_0-\alpha)\bigr) < X_r\ (\Rightarrow c < \alpha-\theta(p_0-\alpha)) \\
   \cline{1-3}
       6&q\ \text{(free)} &   q - \frac{1}{\beta}(p_0 - \alpha) & \text{Perfect competition: }\theta=\theta_r=0: \\
      &&& c = \alpha \\
      &&&  0 < q < X \\
      &&& \frac{1}{\beta} (p_0-\alpha) < q < X_r + \frac{1}{\beta} (p_0-\alpha) \\
   \hline
%%% Case q=X, q_r>0:
% ------------------------
    7&X  & \frac{1}{1+\theta_r} X - \frac{1}{1+\theta_r} \frac{1}{\beta} (p_0-\alpha)
    &  \begin{cases}
            X \leq \frac{1}{\theta\theta_r + \theta + \theta_r} \frac{1}{\beta} \bigl(\alpha - c + \theta_r(p_0-c)\bigr) &\text{if } \neg(\theta=\theta_r=0),\ (\Rightarrow \alpha > c-\theta_r(p_0-c))  \\
           \alpha \geq  c &\text{if } \quad \theta=\theta_r=0
        \end{cases} \\   
    &&&  0 <\frac{1}{1+\theta_r} X - \frac{1}{1+\theta_r} \frac{1}{\beta} (p_0-\alpha) < X_r
    \ (\overset{(q)}{\Rightarrow} c < \alpha - \theta(p_0-\alpha)) \\
   \hline
%%% Case q>0, q_r=X_r:
%--------------------------
 8&\frac{1}{1+\theta} X_r  +\frac{1}{1+\theta} \frac{1}{\beta} (p_0-c) 
    & X_r &  \begin{cases}
           X_r \leq \frac{1}{\theta\theta_r + \theta + \theta_r}\frac{1}{\beta}\bigl(\alpha - c - \theta(p_0-\alpha)\bigr) &\text{if } \neg(\theta=\theta_r=0), \ (\Rightarrow c < \alpha - \theta(p_0-\alpha)) \\
            \alpha \geq c &\text{if } \quad \theta=\theta_r=0
       \end{cases} \\
       &&& 0 < \frac{1}{1+\theta} X_r  +\frac{1}{1+\theta} \frac{1}{\beta} (p_0-c) < X \\
   \hline
%%% Case q=X, q_r=X_r:
%-------------------------
  9&X & X_r & X \leq \frac{1}{\theta\theta_r + \theta + \theta_r} \frac{1}{\beta} \bigl(\alpha - c + \theta_r(p_0-c)\bigr)\ \text{if } \neg(\theta=\theta_r=0)\  (\Rightarrow \alpha > c- \theta_r(p_0-c))  \\
  &&& \ldots \ldots \\
%  &&& X_r \leq \frac{1}{\theta\theta_r + \theta + \theta_r} \frac{1}{\beta} \bigl(\alpha - c - \theta(p_0-\alpha)\bigr) 
% \ \text{if } \neg(\theta=\theta_r=0)\ (\Rightarrow  c < \alpha - \theta(p_0 -\alpha)) \\
%  &&& X \leq \frac{1}{1+\theta} X_r +  \frac{1}{1+\theta} \frac{1}{\beta} (p_0-c) \\
%  &&& X_r \leq \frac{1}{1+\theta_r} X - \frac{1}{1+\theta_r} \frac{1}{\beta} (p_0 - \alpha)\  (\Rightarrow  X > \frac{1}{\beta}(p_0-\alpha)) \\
%%   \hline   
 \end{array} 
 $}
\renewcommand{\arraystretch}{1.0}
\end{table}
\end{frame}



\begin{frame}{Perfect competition; Production and consumption}

\begin{block}{Case: No market power, i.e., perfect competition [Walras, 1890]}
If $\theta=\theta_r=0$, then the integrated player and the separated players have identical market equilibria.
\end{block}
%\begin{block}{Interior solution}
%Let the optimal production $q$ and consumption $q_r$ be not at bounds, that is, we have strict inequalities
%\begin{equation*}
%    0 < q_{} < X_{},\quad 0 < q_r < X_r.
%\end{equation*}    
%Then, in case of the integrated player, it follows $\alpha=c$ for all $\theta\in[0,1]$.
%In case of the separated players,
%we have: $\theta=\theta_r=0 \iff \alpha=c$.
%\end{block}
Note for the following slides:
\begin{itemize}
  \item Production decreases $\to$ price increases
  \item Consumption decreases $\to$ price decreases
\end{itemize}
\begin{block}{Separated player: Production and consumption}
Assuming $\theta_r=\theta$, optimal production and consumption of the separated players monotonically decrease in $\theta \in (0,1]$. 
\end{block}
\end{frame}

\begin{frame}{Integrated player}
\begin{block}{Integrated player: Production and consumption}
The integrated players reduces quantities with increasing CV parameter more on the consumer-side or more on the producer-side, depending on parameters. 
\end{block}
Example:
\begin{block}{Case: Production at capacity ($q=X$), and consumption $q_r$ not at bounds:}
$
\begin{aligned}  
      q_r &= X-\frac{1}{1+\theta} \frac{1}{\beta}(p_0-\alpha) \\
      q-q_r &=\qquad \frac{1}{1+\theta} \frac{1}{\beta}(p_0-\alpha)
     \end{aligned}\qquad\qquad
\scriptsize 
\begin{array}{|c|} 
    \hline
    \text{Parameter region of Case 7} \\ \hline    
    {\begin{aligned} 
             & \alpha \geq c \\
             & \frac{1}{1+\theta} \frac{1}{\beta}(p_0-\alpha) < X < X_r+\frac{1}{1+\theta} \frac{1}{\beta}(p_0-\alpha)
    \end{aligned}} \\ \hline 
\end{array}
$
\end{block}
Hence:
\begin{itemize}
 \item If $p_0 > \alpha$:\quad  $\theta \uparrow 1 \quad \Rightarrow\ q_r$ increases,
       $(q-q_r)$ decreases $ \quad \Rightarrow\ $ Market price rises 
  \item If $p_0 < \alpha$:\quad $\theta \uparrow 1 \quad \Rightarrow\ q_r$ deceases, $(q-q_r)<0$ increases $ \quad \Rightarrow\ $ Market price fall
\end{itemize}

\end{frame}

\begin{frame}{More results (under additional assumptions)}
Assumptions from this slide onwards:
\begin{itemize}
\item `Reasonable' relations between costs  and price intercept $p_0$
\item `Full' market power  
\end{itemize}
\vspace{-2ex}
\begin{equation*}
 \left.
 \begin{aligned} \centering
      c<\alpha<p_0 & \\
      c+p_0<2\alpha& \\
      \theta=\theta_r=1 &
 \end{aligned}
\right\}\ (C)
\end{equation*}
\vspace{-1ex}
\begin{block}{Production, and consumption\ldots}
\ldots of integrated player is higher than of separated players.
\end{block}

%\begin{block}{Prices} 
%The lower bound of price in both the integrated and separated player case is $\frac{1}{2}(p_0+c)$, which is the monopoly price. In the case of the integrated player, the bound is attained in region $(8 \cap 8)$ and $(8 \cap 6)$, whereas for the separated player, the bound is not attained. 
%\end{block}

\begin{block}{Profit\ldots}
of integrated player is higher or equal to the sum of profits of separated players. 
\end{block}

\begin{block}{Negative profits of integrated player}
    In some of the cases, negative profits occcur on either supply or demand side.
\end{block}

\end{frame}


%%% Reg 1
\begin{frame}{Case: Very small production capacity}
% [trim={left bottom right top},clip]
% Ex. 1: trim from left edge
%\vspace{-0.8ex}
(7,4) := (Case 7 of integrated player, Case 4 separated Player)
\begin{figure}
%\fbox{
     \includegraphics[trim={1.3cm 2.7cm 10cm 2.5cm},clip,scale=0.3]{EURO_Reg1}
%}
\end{figure}
%\vspace{-1.7ex}
\begin{itemize}
   \item Very small production capacity $\to$ Productions are at capacity ($q=X$) 
    \item Because $\alpha < p_I = p_S$ $\to$ No consumptions ($q_r=0$)
\end{itemize}
\end{frame}

%%% Reg 2
\begin{frame}{Case: Production capacity gets bigger}
% [trim={left bottom right top},clip]
% Ex. 1: trim from left edge
%\vspace{-1ex}
\begin{figure}
%\fbox{
    \includegraphics[trim={1.3cm 2.7cm 10cm 2.5cm},clip,scale=0.3]{EURO_Reg2}
%}
\end{figure}
%\vspace{-1ex}
\begin{itemize}
   \item Productions still at capacity; separated player still with no consumption
    \item $\alpha < p_S < p_I$ $\to$ To raise prices, integrated player starts (non-profitable) consumption
\end{itemize}
\end{frame}

%%% Reg 3
\begin{frame}{Case: Moderate production capacity }
% [trim={left bottom right top},clip]
% Ex. 1: trim from left edge
%\vspace{-1ex}
\begin{figure}
%\fbox{
     \includegraphics[trim={1.3cm 2.7cm 10cm 2.5cm},clip,scale=0.3]{EURO_Reg3}
%}
\end{figure}
%\vspace{-1ex}
\begin{itemize}
  \item Productions are still at capacity  
   \item Usually $\alpha > p_I \geq p_S$ $\to$ Consumption also for separated player   
\end{itemize}
\end{frame}

%%% Reg 4
\begin{frame}{Case: Small consumption capacity relative to production}
% [trim={left bottom right top},clip]
% Ex. 1: trim from left edge
%\vspace{-1ex}
\begin{figure}
%\fbox{
      \includegraphics[trim={1.3cm 2.7cm 10cm 2.5cm},clip,scale=0.3]{EURO_Reg4}
%}
\end{figure}
%\vspace{-1ex}
\begin{itemize}
    \item Small consumption capacity $\to$ Consumptions are at capacity
     \item Case (8,8): Integrated player behaves as classic monopoly player with monopoly price $p_I=\frac{1}{2}(p_0+c)$
     \item $p_I < p_S < \alpha$: Price higher for separated player, because no incentive to lower prices for the production-only player
\end{itemize}
\end{frame}

%%% Reg 5
\begin{frame}{Case: Large capacities of production and consumption}
% [trim={left bottom right top},clip]
% Ex. 1: trim from left edge
%\vspace{-1ex}
\begin{figure}
%\fbox{
    \includegraphics[trim={1.3cm 2.7cm 10cm 2.5cm},clip,scale=0.3]{EURO_Reg5}
%}
\end{figure}
%\vspace{-1ex}
\begin{itemize}
   \item Separated player: Production and consumption do not depend on bounds
    \item Integrated player: Only difference between production and consumption matters: At least one of production or consumption is at bound
\end{itemize}
\end{frame}



\begin{frame}{Social Welfare: Difference integrated vs. separated}
% [trim={left bottom right top},clip]
% Ex. 1: trim from left edge
\begin{figure}
%\fbox{
    \includegraphics[trim={1.4cm 2.7cm 10cm 2.5cm},clip,scale=0.3]{EURO_SW}
%}
\end{figure}
\begin{itemize}
   \item Red case (7,4): Integrated player raises prices relatively high by (negative-profit) consumption $\to$ moderate profits and low consumer surplus
   \item Green case (8,8): Integrated player consumes at capacity, and `plays undisturbed' with production $\to$ high profits
\end{itemize}
Similar ambiguity results on vertical integration: Salinger (1988), Bushnell et al.\ (2008)
\end{frame}




\begin{frame}{Storage under market power is not clear-cut}

\begin{quote} \structure{
    "The interactions between market power in generation and in storage are complex, suggesting that predictions from one market may not apply elsewhere\ldots"}
    \end{quote} 
 \qquad [Williams \& Green (2023), \textit{Electricity storage and market power}]
\bigskip

\structure{Siohshansi (2014):}
\begin{itemize} 
\item  Considers symmetric two-period producers with market power and own storage, with quadratic production cost (instead of bounds on capacity):
%\begin{itemize}
   \item Storage \structure{may} increase or decrease social welfare compared to the case without storage. 
    \item Gives storage amounts with and without market power [comparison?]
%\end{itemize}
\end{itemize}
\bigskip

\structure{Our simple question:}
\begin{itemize}
\item Under capacity constraints: Do we store more under market power? 
\end{itemize}

\end{frame}



\begin{frame}{Storage under market power}

\begin{itemize}
\item 
(Most) simple analytical model: \structure{Symmetric two-period storage duopoly with: Constant marginal costs, Conjectural variations, Capacity constraints on production and storage, Linear price-demand function}
\end{itemize}
Profit maximization of player (pre-commitment over the time $1$ and $2$):
\begin{align*}
    \max & (q_1-q_r)p_1 + (q_2+\eta q_r) p_2 + c_1 q_1 + c_2 q_2\\
     \text{s.t.}&\left\{
     \begin{aligned}
     & 0\leq q_t \leq X_t,\quad t=1,2,\\
      & 0\leq q_r \leq Y\\
     \end{aligned}\right.
\end{align*}
\begin{itemize} 
  \item $s$: storage amount, $\eta$ efficiency
  \item $p_1 := p_{01} - \beta_1 2 (q_1-q_r)$; $p_2 := p_{02} - \beta_2 2 (q_2+\eta q_r)$
  \item  Conjectural variation: $\frac{dp_t}{dq_t} := -\theta \beta_t$,
      $\frac{dp_1}{dq_r} := \theta \beta_1$, $\frac{dp_2}{dq_r} := -\eta \theta \beta_1$,
    \item 3 variables at low / upper / not at bound: $3\cdot3\cdot3=27$ cases!
\end{itemize}  
\end{frame}

\begin{frame}{Do we store more or less under market power?}
\begin{block}{Interior solution of ($q_1$,$q_r$,$q_2$):}
No interior solution if efficiency $\eta\neq\frac{c_1}{c_2}$\quad [1-period duopoly has interior solution]
\end{block}
\begin{columns}
 
     \column[c]{.7\textwidth}
\begin{block}{Case: 1st-period production at capacity:}
$
\renewcommand{\arraystretch}{.7}
\footnotesize 
\begin{aligned}  
      q_1  &= X_1 \quad (\text{net prod.}=X_1-s)\\
     q_r & = X_1-\frac{1}{2+\theta} \frac{1}{\beta_1}(p_{01}-\eta c_2) \\
     q_2 + \eta q_r  & = \frac{1}{2+\theta} \frac{1}{\beta_2}(p_{02}-c_2) 
     \end{aligned}\qquad\
\scriptsize 
\begin{array}{|c|} 
    \hline
    \text{Parameter region of case} \\ \hline    
    {\begin{aligned} 
             & \eta \geq \frac{c_1}{c_2} \\
             & 0 < q_2 < X_2,\ 0 < q_r < Y 
    \end{aligned}} \\ \hline 
\end{array}
$
\end{block}

\column[c]{.2\textwidth}
If $p_{01}>\eta c_2$:\\
\qquad $\theta\uparrow \implies q_r\uparrow$
\end{columns} 

\vspace{-1ex}

\begin{columns}
  \column[c]{.7\linewidth}
\begin{block}{Case: 2nd-period production at capacity:}
$
\footnotesize 
\begin{aligned}  
      q_1 - q_r  &= \frac{1}{2+\theta} \frac{1}{\beta_1}(p_{01}-c_1)\\
       q_r & = \frac{1}{\eta}\biggl(\frac{1}{2+\theta} \frac{1}{\beta_2}(p_{02}-\frac{c_1}{\eta})-X_2\biggr) \\
     q_2  & = X_2 
     \end{aligned}\quad
\scriptsize      
\begin{array}{|c|l|} 
\hline     
 \text{Parameter region of case} \\ \hline    
        {\begin{aligned} 
             & \eta \leq \frac{c_1}{c_2} \\
             & 0 < q_1 < X_1,\ 0 < q_r < Y 
    \end{aligned}} \\ \hline 
\end{array}
\renewcommand{\arraystretch}{1}
$  
\end{block}

\column[c]{.2\linewidth}
If $p_{02}>\frac{c_1}{\eta}$:\\[1ex]
\qquad $\theta\uparrow \implies s\downarrow$
\end{columns}

\end{frame}

\begin{frame}{Conclusions}
\textbf{Integrated versus Separated producer\&consumer under capacity constraints:} In the case of $\theta=1$ and `reasonable' assumptions on parameters: 
\begin{itemize}
 \item \structure{Production, consumption}, and \structure{profits} of integrated player are higher than separated players.
  \item \structure{Social welfare} and prices are ambiguous between integrated and separated player; depends on proportion of capacities of production and consumption
\end{itemize}
\textbf{Symmetric two-period storage duopoly under capacity constraints:}
\begin{itemize}
    \item With more market power (CV parameter $\theta\uparrow1$), storage amount can in- or decrease 
\end{itemize}
\bigskip
\textbf{Extensions} [may require use of algebraic software]:
\begin{itemize}
    \item Stackberberg (i.e.\ non-precommitment) behavior
    \item n-players, quadratic costs, remove additional assumption, etc.
\end{itemize}

\end{frame}

\begin{frame}{}
\end{frame}


\begin{frame}{Columns and Blocks}

\setbeamercolor{mycolor}{bg=gray}

You can start text high with a top-aligned frame in beamer

Columns:
  \begin{columns}
    \column[T]{.5\linewidth}
      \begin{block}{Title of block}
        Top-aligned block in first column
      \end{block}
      \begin{example}{Title of example block}
        Block in first column
      \end{example}
      Text outside of block in first column
      \vspace{1ex}
      
      \begin{beamercolorbox}[wd=\linewidth, colsep*=4pt]{mycolor}
         This is a box (without title)
       \end{beamercolorbox}
       
   \column[T]{.4\linewidth}
      \begin{block}{Title in block}
        Block in second column 
      \end{block}
      \begin{alertblock}{Title of alert block}
        Block in second column
      \end{alertblock}
      \begin{block}{}
        Block with empty title
      \end{block}
    \end{columns}
    
    \vspace{1ex}
    \begin{beamercolorbox}[wd=\linewidth, colsep*=4pt]{mycolor}
      This is a box (without title)
    \end{beamercolorbox}
 \end{frame}
 




\end{document}
