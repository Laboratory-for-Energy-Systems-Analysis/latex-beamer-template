% Compile with LuaTex or XeTex, because you need the Aptos-Font
% If you use Overleaf, you have to change the compiler (https://www.overleaf.com/learn/how-to/Changing_compiler)

% The folder "Aptos"  with the Aptos-font is required if you do not use a Windows PC, for example by using the online LaTeX-editor Overleaf.

% Things where you have to adapt are with %%%

\documentclass[aspectratio=169,t]{beamer}  % t = top aligned text on slides as in Powerpoint

\usepackage{graphicx} 
\usepackage{amsmath, amssymb}
\usepackage{xcolor} % only required for convenience RGB color definitions further below: [0,255] as in Powerpoint
\usepackage[no-math]{fontspec}

\setsansfont[Path = ./Aptos/,
UprightFont = Aptos,
BoldFont = Aptos-Bold,
ItalicFont = Aptos-Italic,
BoldItalicFont = Aptos-Bold-Italic,
Extension = .ttf]{aptos}

%%% Convenience definition: The center name is used  just only once, in the footnote coming next
\newcommand{\PSIcenter}{PSI Center for Energy and Environmental Sciences}

\setbeamertemplate{footline}[text line]{%
  \parbox{\linewidth}{\vspace*{-8pt}\color{gray}\insertpagenumber\hfill\insertshortauthor:\ \insertshorttitle\hfill\PSIcenter\hfill \insertdate
}}

\setbeamertemplate{navigation symbols}{}

%%% if you use another cloud not found here, copy the corresponding picture out from the Powerpoint-Master-slide:
\pgfdeclareimage[height=\paperheight]{mybackground}{%
PSI_Cloud_Blue}
%PSI_Cloud_Red}
%PSI_Cloud_DarkBlue}

%%% if you use another center not found here, copy the corresponding picture out from the Powerpoint-Master-slide:
\pgfdeclareimage[height=0.5cm]{myCenter}{%
Center_CEE_White}
%Center_CAS_White}
%Center_CCS_White}
%Center_CLS_White}
%Center_CNM_White}
%Center_CPS_White}
%Center_NES_White}
%Center_SCD_White}

% The next two are the same pictures for all centers:
\pgfdeclareimage[height=0.5cm]{mySlideLogo}{PSI_Logo_black}
\pgfdeclareimage[height=0.8cm]{myLogo}{PSI_Logo_White}

\setbeamercolor{title}{fg=white}
\setbeamerfont{title}{size=\fontsize{22}{24}\selectfont}
\setbeamerfont{author}{size=\fontsize{9}{12}\selectfont}
\setbeamerfont{institute}{size=\fontsize{9}{12}\selectfont}
\setbeamerfont{date}{size=\fontsize{9}{12}\selectfont}
\setbeamertemplate{frametitle}{\vspace*{0.7ex}\color{black}\bfseries\insertframetitle}


\setbeamertemplate{title page}{
        \begin{picture}(0,0)
            \put(-28,-154.3){%
                \pgfuseimage{mybackground}}
                
             \put(0, 70){%
                \pgfuseimage{myLogo}}
                
             \put(60, 74){%
                \pgfuseimage{myCenter}}
                
            \put(0,-110.7){%
                \begin{minipage}[b][45mm][t]{226mm}
                   \begin{beamercolorbox}[wd=0.9\textwidth,sep=1em]{title}
                       \usebeamerfont{title}\textbf{\inserttitle}\vspace*{0.5ex}\par
                       \usebeamerfont{subtitle}{\textbf{\insertsubtitle}}
                    \end{beamercolorbox}       
               \end{minipage}}

             \put(0,-240){%
                \begin{minipage}[b][45mm][t]{226mm}
                   \begin{beamercolorbox}[wd=0.9\textwidth,sep=1em]{title}
                       \usebeamerfont{author}{\insertauthor\par}
                        \usebeamerfont{institute}{\insertinstitute},
                        \usebeamerfont{date}{\insertdate}  
                    \end{beamercolorbox}       
               \end{minipage}}

            \end{picture}
}

\logo{\pgfuseimage{mySlideLogo}\hspace{22pt}\vspace{225pt}}

% make the bullet smaller:
\setbeamertemplate{itemize item}{\small\raise0.5pt\hbox{\color{black}\textbullet}}
\setbeamertemplate{itemize subitem}{\small\raise0.5pt\hbox{\color{black}\textbullet}}
\setbeamertemplate{itemize subsubitem}{--}
% keep subitems at normal size:
\setbeamertemplate{itemize/enumerate subbody begin}{\normalsize}
\setbeamertemplate{itemize/enumerate subsubbody begin}{\normalsize}
% remove first-level indent:
\settowidth{\leftmargini}{\usebeamertemplate{itemize item}}
\addtolength{\leftmargini}{\labelsep}

% some colors for convenience
\definecolor{PSIblue}{RGB}{0,20,230}
\definecolor{PSIgreen}{RGB}{0,240,160}
\definecolor{PSIred}{RGB}{220,0,90}
\definecolor{PSIviolet}{RGB}{110,20,220}
\definecolor{PSIyellow}{RGB}{240,245,0}
\definecolor{PSIpink}{RGB}{240,80,250}

% modify blocks to your likening 
\setbeamertemplate{blocks}[rounded][shadow=false]
\setbeamercolor{block title}{bg=PSIblue,fg=white}
\setbeamercolor{block body}{bg=PSIblue!10,fg=black}
\setbeamercolor{block title example}{bg=PSIgreen,fg=black}
\setbeamercolor{block body example}{bg=PSIgreen!10,fg=black}
\setbeamercolor{block title alerted}{bg=PSIred,fg=white}
\setbeamercolor{block body alerted}{bg=PSIred!10,fg=black}
\setbeamercolor{structure}{fg=PSIblue}
\setbeamercolor{alert}{fg=PSIred}

% Disable shading between block title and block content
\makeatletter
\pgfdeclareverticalshading[lower.bg,upper.bg]{bmb@transition}{200cm}{color(0pt)=(lower.bg); color(4pt)=(lower.bg); color(4pt)=(upper.bg)}
\makeatother

%% My personal suggestions:

% Remove the silly "Figure" in front of a figure caption in beamer:
\setbeamertemplate{caption}{\raggedright\insertcaption\par}

% The advice to use non-serif fonts for math was from a time
% where screen-resolution was low. With HD and UHD, this is obsolete IMHO.
\usefonttheme[onlymath]{serif}
% The math spacing is wrong. We have to reset the letters font to T1 
\AtBeginDocument{\DeclareSymbolFont{pureletters}{T1}{lmr}{\mddefault}{it}}


%%% Start of presentation

\title[Short title]{Add Title of\\Presentation}
\subtitle{Add Subtitle of\\Presentation}
\author[Author in Footnote]{Author on Title Page}
\date{28.06.2024}
\institute{Occation of Presentation}


\begin{document}

\begin{frame}[plain,c] % c = centered (or you have shift coordinates in above code)
 \titlepage
\end{frame}


\begin{frame}{Example slide title}

Formula in serif-font as in papers (can be changed in LaTeX preamble):
\begin{equation}
    a + b = c \int f(x)\,dx
\end{equation}

\begin{block}{Remember this!}
The block appearance can be easily changed in the LaTeX preamble
\end{block}

\begin{alertblock}{Very Important}
Sample text in red box
\end{alertblock}

\begin{exampleblock}{Top algined}
In the LaTeX preamble, we set [t] for top aligned content as in Powerpoint 
\end{exampleblock}

\end{frame}


\begin{frame}{Slide title}

Bullets are smaller than in LaTeX, and begin aligned with text: 
\begin{itemize}
 \item Level 1
 \begin{itemize}
     \item Level 2
     \begin{itemize}
          \item Level 3
      \end{itemize}    
 \end{itemize}     
\end{itemize}

\end{frame}



\begin{frame}{A list that is revealed}

  \begin{itemize}
  \item PSI-style 
  \item[--]  With dash instead of bullet if you like
    \begin{enumerate}
    \item Enumerated 1a
    \item Enumerated 1b
    \end{enumerate}
    \pause
  \item Item 2 (revealed on next slide)
    \pause
  \item Item 3
    \begin{itemize}
    \item Item 4
      \begin{itemize}
      \item Item 5
      \end{itemize}
    \end{itemize}
  \end{itemize}

\end{frame}



\begin{frame}{The PSI colors}

\textcolor{PSIgreen}{\rule{2cm}{1cm}}
\textcolor{PSIyellow}{\rule{2cm}{1cm}}
\textcolor{PSIpink}{\rule{2cm}{1cm}}
\textcolor{PSIblue}{\rule{2cm}{1cm}}
\textcolor{PSIred}{\rule{2cm}{1cm}}
\textcolor{PSIviolet}{\rule{2cm}{1cm}}

 \structure{This is a color predefined in beamer called \texttt{structure}}

\end{frame}

\begin{frame}[squeeze]{Titles over two lines\\ are possible}
  Some formulas, which can be tagged, e.g.\ \thetag{$\ast$}:
  \begin{align*}
    \Gamma  & = \int_a^b f(x)\,dx             && \tag{$\ast$}\\
    \sum_{n=1}^N x_{i_n} &= \Bigl(\frac{a+b}{c+d}\Bigr)\cdot \sin(x)  &&
  \end{align*}
  \begin{itemize}
  \item  $\alpha = \zeta$; \texttt{\textbackslash boldsymbol}: $\boldsymbol{\alpha} = \boldsymbol{\zeta}$ 
  \item $\mathcal{F}_t$, $\mathbb{R}^n$
  \item $\{t \mid t=1,\dots,T\}$
  \item $f\colon X\to Y$; $f\colon x\mapsto y$; use \texttt{\textbackslash colon}
  \item $y_nx_n^m$ versus $y_n^{}x_n^m$
  \end{itemize}
   \alert{This is a frame with the usual `squeeze' option of beamer to narrow linespacing a little bit; the color of this text is called \texttt{alert} in beamer}
  
\end{frame}

\begin{frame}{Columns and Blocks}

\setbeamercolor{mycolor}{bg=gray!30}

You can start text high with a top-aligned frame in beamer

Columns:
  \begin{columns}
    \column[T]{.5\linewidth}
      \begin{block}{Title of block}
        Top-aligned block in first column
      \end{block}
      \begin{example}{Title of example block}
        Block in first column
      \end{example}
      Text outside of block in first column
      \vspace{1ex}
      
      \begin{beamercolorbox}[wd=\linewidth, colsep*=4pt]{mycolor}
         This is a box (without title)
       \end{beamercolorbox}
       
   \column[T]{.4\linewidth}
      \begin{block}{Title in block}
        Block in second column 
      \end{block}
      \begin{alertblock}{Title of alert block}
        Block in second column
      \end{alertblock}
      \begin{block}{}
        Block with empty title
      \end{block}
    \end{columns}
    
    \vspace{1ex}
    \begin{beamercolorbox}[wd=\linewidth, colsep*=4pt]{mycolor}
      This is a box (without title)
    \end{beamercolorbox}
 \end{frame}
 

 \begin{frame}[c]{Figures, Centering}
 A slide with vertical centering.
 
  \begin{figure}
    \includegraphics[width =.5\textwidth]{PSI_Logo_black}
    \caption{Figures are automatically horiz.\ centered in beamer}
    \label{fig:PSI}
  \end{figure}
  
\end{frame}

\begin{frame}{Overwrite the margins}
  You can overwrite the margins with a wide column:
  \setbeamercolor{mycolor}{fg=PSIblue, bg=structure!10}
\begin{columns}
  \column{1.1\linewidth}
    We need more space:
  \begin{beamercolorbox}[wd=\textwidth, dp=0.7\textheight, colsep*=4pt]{mycolor}
    This is a very long text that needs a lot of space this is a very long text that needs a lot of space this is a very long text that needs a lot of space
  \end{beamercolorbox}
\end{columns}
\end{frame}



\end{document}
